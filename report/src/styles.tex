%%% Размер полей %%%
\geometry{a4paper,top=2cm,bottom=2cm,left=1.5cm,right=1.5cm}

%%% Кодировки и шрифты %%%
%\renewcommand{\rmdefault}{ftm} % Включаем Times New Roman

%%% Выравнивание и переносы %%%
\sloppy					% Избавляемся от переполнений
\clubpenalty=10000		% Запрещаем разрыв страницы после первой строки абзаца
\widowpenalty=10000		% Запрещаем разрыв страницы после последней строки абзаца

%%% Библиография %%%
\makeatletter
\bibliographystyle{utf8gost705u}	% Оформляем библиографию в соответствии с ГОСТ 7.0.5
\renewcommand{\@biblabel}[1]{#1.}	% Заменяем библиографию с квадратных скобок на точку:
\makeatother

%%% Цвета гиперссылок %%%
\definecolor{linkcolor}{rgb}{0.9,0,0}
\definecolor{citecolor}{rgb}{0,0.6,0}
\definecolor{urlcolor}{rgb}{0,0,1}
\hypersetup{
    colorlinks, linkcolor={linkcolor},
    citecolor={citecolor}, urlcolor={urlcolor}
}

%%% Оглавление %%%
\renewcommand{\cftpartleader}{\cftdotfill{\cftdotsep}} % делает линии из точек для parts
\renewcommand{\cftchapleader}{\cftdotfill{\cftdotsep}} % делает линии из точек для chapters
\usepackage{hyperref} % Кликабельные ссылки в оглавлении
